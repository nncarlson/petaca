\documentclass[11pt]{article}

\usepackage[latin1]{inputenc}
\usepackage{times}
\usepackage[T1]{fontenc}

\usepackage{underscore}
\usepackage{verbatim}
\usepackage{enumitem}

\usepackage[nofancy]{latex2man}

\usepackage[margin=1.25in,letterpaper]{geometry}

\setlength{\parindent}{0pt}

\begin{document}

%%% SET THE DATE
\setDate{August 2013}
\setVersion{1.0}

\begin{Name}{3}{Secure Hashes}{Neil N. Carlson}{Petaca}{Secure Hashes}
%%% THE ABSTRACT GOES HERE
This collection of modules provides a common interface to several different
secure hash or message digest algorithms.  Currently only MD5 and SHA1 are
implemented, but others are easily implemented within the provided framework.
\end{Name}

\section{Synopsis}
\begin{description}[style=nextline]
\item[Usage]
  \verb+use :: map_any_type+
\item[Derived Types]
  \texttt{map_any},\texttt{ map_any_iterator}
\item[Parameters]
\end{description}

\section{Mo}

module secure_hash_class defines the abstract base class secure_hash
from which specific secure hash algorithms are derived.  It has the
methods update, hexdigest, reset.

instantiation.


m = md5_hash()

\section{SECTION TITLE}

\subsection{Type bound subroutines}
\begin{description}[style=nextline]\setlength{\itemsep}{0pt}
\item[\texttt{some_subroutine(and_its_arguments)}]
\end{description}

\subsection{Type bound functions}
\begin{description}[style=nextline]\setlength{\itemsep}{0pt}
\item[\texttt{some_function(and_its_arguments)}]
\end{description}

\section{Example}
\begin{verbatim}
%%% SOME EXAMPLE CODE
\end{verbatim}

\section{Bugs}
Bug reports and improvement suggestions should be directed to
\Email{neil.n.carlson@gmail.com}

\LatexManEnd

\end{document}
